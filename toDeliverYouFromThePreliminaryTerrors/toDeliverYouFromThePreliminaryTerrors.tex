\documentclass{ximera}

\renewcommand{\d}{\mathop{}\!d}


\title{To deliver you from the preliminary terrors}

\begin{document}
\begin{abstract}
\end{abstract}
\maketitle

The preliminary terror, which chokes off most fifth-form boys from
even attempting to learn how to calculate, can be abolished once for
all by simply stating what is the meaning---in common-sense terms---of the
two principal symbols that are used in calculating.

These dreadful symbols are:

\begin{enumerate}
\item $d$ which merely means ``a little bit of.''

Thus $dx$ means a little bit of $x$; or $du$ means a little bit of
$u$. Ordinary mathematicians think it more polite to say ``an element
of,'' instead of ``a little bit of.'' Just as you please. But you will
find that these little bits (or elements) may be considered to be
indefinitely small.

\item $\int$ which is merely a long S, and may be called (if you like)
  ``the sum of.''

Thus $\int \d x$ means the sum of all the little bits of $x$; or $\int
\d t$ means the sum of all the little bits of $t$. Ordinary
mathematicians call this symbol ``the integral of.'' Now any fool can
see that if $x$ is considered as made up of a lot of little bits, each
of which is called $\d x$, if you add them all up together you get the
sum of all the $\d x$'s, (which is the same thing as the whole of
$x$). The word ``integral'' simply means ``the whole.'' If you think
of the duration of time for one hour, you may (if you like) think of
it as cut up into $3600$ little bits called seconds. The whole of the
$3600$ little bits added up together make one hour.

When you see an expression that begins with this terrifying symbol,
you will henceforth know that it is put there merely to give you
instructions that you are now to perform the operation (if you can) of
totaling up all the little bits that are indicated by the symbols
that follow.
\end{enumerate}
That's all. 

\end{document}

